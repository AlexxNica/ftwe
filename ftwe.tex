\documentstyle[12pt]{article}

\begin{document}

WHat the theorem says: If the market is at equilibrium, then the
allocation of goods must be Pareto optimal.


What does it mean to be Pareto Optimal?  The intuition is that if I
change my choices about how to participate in the market economy then
I can't increase I'll end up worse off --- I'll lower my utility.

How can we formalize how we participate in the economy?  Let's build a
model.  We'll split the world up into two parts: consumers and firms.



\end{document}